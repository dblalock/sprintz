
% \documentclass[conference]{IEEEtran}
% \documentclass{sig-alternate} % pre 2017
% \documentclass[sigconf]{acmart}  % starting in 2017
\documentclass{vldb}
\usepackage{balance}  % vldb requirement; for  \balance command on last page
%-------------------------------------------------------------- Includes

\usepackage{bbm}  % who knows?

\usepackage{amsmath}          % basic math
\usepackage{amssymb} 			    % math symbols
% \usepackage{amsthm}           % theorems
\usepackage{textcomp}

% \usepackage{float}            % make figures work
% \usepackage{cite}             % citations would be nice
\usepackage{url}              % better urls; magically keeping doc from breaking due to urls in refs

% \usepackage{tabu}
\usepackage{array}            % multiline table cells; somehow
\usepackage{tabularx}         % better tables
% \usepackage[table]{xcolor}    % colored table cells  % incompatible with sigconf
\usepackage{colortbl}
\newcolumntype{Y}{>{\centering\arraybackslash}X}	% centered column type for tabularx

% stuff for piecewise functions
\usepackage{mathtools}          %loads amsmath as well
\DeclarePairedDelimiter\Floor\lfloor\rfloor
\DeclarePairedDelimiter\Ceil\lceil\rceil

\DeclareMathOperator*{\argmin}{arg\,min} % argmin
\DeclareMathOperator*{\argmax}{arg\,max} % argmax

\usepackage{pbox}   % for trick to force linebreaks in table cells

\usepackage{setspace}
% \usepackage{setspace}

% \usepackage{booktabs} % acm recommended, and for pandas latex tables

%-------------------------------------------------------------- Algorithm setup

\usepackage{algorithm}
\usepackage[noend]{algpseudocode} % I think this removes trailing "end {if,for,while}"

% \algnewcommand{\LineComment}[1]{\State \(\triangleright\) #1} % left-aligned comments
% \algnewcommand{\SideComment}[1]{\(//\) #1}
\algnewcommand{\COMMENT}[2][.5\linewidth]{\leavevmode\hfill\makebox[#1][l]{//~#2}}
\algnewcommand{\LineComment}[1]{\State \(//\) #1}	% left-aligned comments
\algnewcommand\RETURN{\State \textbf{return} }


%-------------------------------------------------------------- Figures setup

% \usepackage[pdftex]{graphicx}
\usepackage{graphicx}
\usepackage[space]{grffile}   % allow spaces in file names
% declare the path(s) where your graphic files are
\graphicspath{{../figs/}}
% and their extensions so you won't have to specify these
\DeclareGraphicsExtensions{.pdf,.jpeg,.jpg,.png}


%\textfloatsep: space between last top float or first bottom float and the text (default = 20.0pt plus 2.0pt minus 4.0pt).
%\intextsep : space left on top and bottom of an in-text float (default = 12.0pt plus 2.0pt minus 2.0pt).
\setlength{\textfloatsep}{4pt}
\setlength{\intextsep}{4pt}

% \usepackage{caption}
% \usepackage[font={small,it}]{caption}
\usepackage[font={bf}]{caption}
% \setlength{\abovecaptionskip}{-2pt} % less space between captions and figures
\setlength{\abovecaptionskip}{-5pt}	% less space between captions and figures
% \setlength{\belowcaptionskip}{-13pt}  % less space below captions
% \setlength{\belowcaptionskip}{-10pt}  % less space below captions
\setlength{\belowcaptionskip}{-4pt}	% less space below captions

\usepackage{paralist}
\setdefaultleftmargin{10pt}{10pt}{}{}{}{}

% \usepackage{changepage}
% \usepackage{tabulary}

\usepackage{outlines}
\usepackage{enumitem}
\newcommand{\ItemSpacing}{0mm}
\newcommand{\ParSpacing}{0mm}
\setenumerate[1]{itemsep={\ItemSpacing},parsep={\ParSpacing},label=\arabic*.}
% \setenumerate[2]{itemsep={\ItemSpacing},parsep={\ParSpacing},label=\arabic*.}
\setenumerate[2]{itemsep={\ItemSpacing},parsep={\ParSpacing}}

\usepackage[linewidth=1pt]{mdframed}
\mdfsetup{frametitlealignment=\center, skipabove=0, innertopmargin=1mm,
innerleftmargin=2mm, leftmargin=0mm, rightmargin=0mm}

%-------------------------------------------------------------- Miscellaneous setup

\newlength\myindent
\setlength\myindent{2em}
\newcommand\bindent{%
  \begingroup
  \setlength{\itemindent}{\myindent}
  \addtolength{\algorithmicindent}{\myindent}
}
\newcommand\eindent{\endgroup}

% remove unwanted space between paragraphs;
% it's set by the IEEE conference format, but no papers from this conference have it
% \parskip 0ex plus 0.2ex minus 0.1ex

% make vectors be bold instead of with arrows
\renewcommand{\vec}[1]{\mathbf{#1}}
% add 'mat' command to make matrices bold
\newcommand{\mat}[1]{\mathbf{#1}}

\DeclarePairedDelimiter\ceil{\lceil}{\rceil}
\DeclarePairedDelimiter\floor{\lfloor}{\rfloor}

\newtheorem{Definition}{Definition}[section]

% ------------------------ convenience commands
\newcommand\eps\varepsilon
\DeclareMathOperator{\infimum}{inf}
\renewcommand\inf\infty

\renewcommand{\c}{\vec{c}}
\newcommand{\q}{\vec{q}}
\renewcommand{\r}{\vec{r}}
\newcommand{\x}{\vec{x}}
\newcommand{\xhat}{\hat{\vec{x}}}
\newcommand{\y}{\vec{y}}
\newcommand{\yhat}{\hat{\vec{y}}}
\newcommand{\z}{\vec{z}}
\newcommand{\zhat}{\hat{\vec{z}}}

\newcommand{\R}{\mathbb{R}}

\newcommand{\onehalf}{\frac{1}{2}}

\newcommand{\pie}[1]{\frac{\pi}{#1}}
% \newcommand{\pitwo}{\frac{\pi}{2}}
% \newcommand{\pifour}{\frac{\pi}{4}}

% \newcommand{\norm}[1]{\left\lVert #1 \right\rVert}
\DeclarePairedDelimiter\abs{\lvert}{\rvert}%
\DeclarePairedDelimiter\norm{\lVert}{\rVert}%

\DeclareMathOperator{\Beta}{Beta}
\DeclareMathOperator{\Normal}{\mathcal{N}}
\DeclareMathOperator{\erf}{erf}
\DeclareMathOperator{\Var}{Var}

% make *all* text 10pt
% \renewcommand{\footnotesize}{\normalsize}
% \renewcommand{\footnotesize}{\small}
% \renewcommand{\small}{\normalsize}


\usepackage[all]{nowidow}

\newcommand{\mine}{\textsc{Sprintz}}
\newcommand{\minesp}{\textsc{Sprintz}\text{ }}
\newcommand{\fire}{\textsc{Fire}\text{ }}
\newcommand{\justfire}{\textsc{Fire}}

\begin{document}

% \setcopyright{rightsretained}

%Conference
% \copyrightyear{2017}
% \acmYear{2017}
% \setcopyright{acmlicensed}
% \acmConference{KDD '17}{August 13--17, 2017}{Halifax, NS, Canada}\acmPrice{15.00}\acmDOI{10.1145/3097983.3098195}
% \acmConference{KDD'17}{}{August 13--17, 2017, Halifax, NS, Canada.} \acmPrice{15.00} \acmDOI{10.1145/3097983.3098195}
% \acmISBN{978-1-4503-4887-4/17/08}
% \acmISBN{ISBN 978-1-4503-4887-4/17/08}

% \fancyhead{}  % apparently KDD printer wants no headers so can add their own
\vldbTitle{Sprintz: Time Series Compression for the Internet of Things}
\vldbDOI{https://doi.org/TBD}

% ================================================================
\title{Sprintz: Time Series Compression \\ for the Internet of Things}
% ================================================================

% \author{Davis W. Blalock}
% \affiliation{
%   \institution{Computer Science and Artificial \\ Intelligence Laboratory}
%   \institution{Massachusetts Institute of Technology}
% }
% \email{dblalock@mit.edu}

% \author{John V. Guttag}
% \affiliation{
%   \institution{Computer Science and Artificial \\ Intelligence Laboratory}
%   \institution{Massachusetts Institute of Technology}
% }
% \email{guttag@mit.edu}

\numberofauthors{3}

\vldbAuthors{Davis Blalock, Sam Madden, John Guttag}

\author{
    \alignauthor
    Davis Blalock    \\
    \affaddr{Computer Science and Artificial Intelligence Laboratory}\\
    % \affaddr{Massachusetts Institute of Technology} \\
    \affaddr{MIT} \\
    \email{dblalock@mit.edu}
    \alignauthor
    Sam Madden    \\
    \affaddr{Computer Science and Artificial Intelligence Laboratory}\\
    % \affaddr{Massachusetts Institute of Technology} \\
    \affaddr{MIT} \\
    \email{madden@csail.mit.edu}
    \alignauthor
    John Guttag    \\
    \affaddr{Computer Science and Artificial Intelligence Laboratory}\\
    % \affaddr{Massachusetts Institute of Technology} \\
    \affaddr{MIT} \\
    \email{guttag@csail.mit.edu}
}

\maketitle

% ------------------------------------------------
\begin{abstract}
% ------------------------------------------------

% By 2020, mobile and connected devices will be produce over \%d\% of the world's data. Much of

 % to the greatest extent possible. % Many time series databases propose some method of doing this, but, to the best of our knowledge, there is neither a systematic assessment of what approaches are most effective nor a compression algorithm that clearly outperforms others on such data. % Unfortunately, existing compression algorithms either provide only limited compression when applied to time series, degrade data quality, or are not suitable for execution on low-power processors.

Due to the rapid proliferation of connected devices, sensor-generated time series constitute a large and growing portion of the world's data. In order to reduce storage, power, and transmission costs, it is desirable to compress this data.

We introduce a time series compression algorithm that achieves state-of-the-art compression ratios and up to twice the speed of its closest rivals. Moreover, because our approach requires $<$1KB of memory, it can be run on low-power sensors at the network edge, offloading computation from the central database and reducing network load. A key component of our method is a high-speed forecasting algorithm that can be trained online and significantly outperforms alternatives such as delta coding.

Extensive experiments on datasets from many domains show that these results hold not only for sensor data but also across a wide array of other time series.

%, our primary application of interest, but also across a wide array of other time series.

% in addition to our own.

% but also accelerates many queries of practical interest relative to existing compression schemes.

% We also provide an open-source benchmark library that can be used to assess dozens of different queries and compressors on various datasets.

 % Our approach also outperforms even domain-specific algorithms designed for more powerful hardware in some cases.

\end{abstract}

% ------------------------------------------------
% CCS taxonomy stuff / keywords
% ------------------------------------------------


% \begin{CCSXML}
% <ccs2012>
% <concept>
% <concept_id>10003752.10003809.10010031.10002975</concept_id>
% <concept_desc>Theory of computation~Data compression</concept_desc>
% <concept_significance>500</concept_significance>
% </concept>
% <concept>
% <concept_id>10003752.10003809.10010047</concept_id>
% <concept_desc>Theory of computation~Online algorithms</concept_desc>
% <concept_significance>300</concept_significance>
% </concept>
% <concept>
% <concept_id>10010583.10010588.10003247</concept_id>
% <concept_desc>Hardware~Signal processing systems</concept_desc>
% <concept_significance>100</concept_significance>
% </concept>
% </ccs2012>
% \end{CCSXML}

% \ccsdesc[500]{Theory of computation~Data compression}
% \ccsdesc[300]{Theory of computation~Online algorithms}
% \ccsdesc[100]{Hardware~Signal processing systems}

% \keywords{Compression, Time Series, Performance}

% ================================================================
\section{Introduction} \label{sec:intro}
% ================================================================


% Time series are ubiquitous and only growing in importance thanks to the proliferation of sensor data from autonomous vehicles, smart phones, wearables, and other connected devices. Although a huge volume of this data is stored in modern databases---many of them designed specifically for time series []---there has been relatively little work on how best to compress this data.

% Alternative 1st paragraph:

% Thanks to the exponentially increasing [] number of embedded, mobile, and wearable devices producing sensor data, there is a rapidly growing need to store time series. This has given rise to numerous time series databases to manage and query this data, both in industry [] and academia [].

% In order to store the data efficiently, all of these systems require some form of compression algorithm.

Time series are ubiquitous and growing in importance thanks to the proliferation of sensor data from autonomous vehicles, smart phones, wearables, and other connected devices. Although a huge volume of this data is currently stored in databases---many of them designed specifically for time series []---there has been relatively little work on how best to compress this data.

At present, systems typically use general-purpose compression algorithms such as LZ4 [], generic floating point compression algorithms [], or generic integer compression algorithms [], possibly with some form of invertible preprocessing [].

These methods are effective to some extent, but as we show experimentally, there is considerable room for improvement. Moreover, because of the nature of time series data and workloads, there are a number of desirable attributes for a time series compression algorithm that these approaches often lack. In particular:

% Because of the nature of time series data and workloads, an ideal compression algorithm would have the following characteristics, in addition to high compression ratio:

% Unfortunately, each of these approaches has drawbacks. Moreoever, as we show experimentally, it is possible to achieve a much better speed-compression tradeoff than what current methods offer.

% In addition to high compression ratio, it is desirable for a time series compression algorithm to have the following properties:

\begin{enumerate}
\itemsep0mm
\item \b{Support for fast scans}. Time series workloads are not only read-heavy [], but often necessitate scans through the data []. This means not only that high decompression speed is essential, but that it is desirable to push down queries directly to the compressed representation.
\item \b{Small block size}. To accelerate queries over arbitrary time intervals, many databases use indexing structures that partition time series into many small sections []. A compression algorithm must therefore be effective even on small numbers ($<$100) of samples.
\item \b{Low memory}. To reduce network usage and offload computation to clients, it is desirable for the compressor to run on clients []. Unfortunately, clients producing time series data are often low-power sensors with only a few KB of RAM [].
% \item \b{High decompression speed}. While compression speed is also desirable, time series workloads are often read-heavy [] or even append-only [], meaning that decompression will run many more times than compression. Moreover, as mentioned above, compression can often be carried out at the client.
\item \b{Exploit related columns}. Time series often consist of several signals that will almost always be accessed together. For example, one would rarely access only X-axis acceleration without Y-axis and Z-axis, or longitude without latitude. It is desirable to make accessing related columns together inexpensive.
\item \b{Amenable to low-bitwidth integer data}. Almost any time series reflecting a real-world signal will be digitized using an Analog-to-Digital Converter (ADC). This means that the data can be represented as integers of at most 32 bits \cite{digikeyADCs}, and typically 16 or fewer. For example, even lossless audio codecs store only 16 bits \cite{flac, shorten}. % Furthermore, there is empirical evidence that most time series can be quantized to 6 or fewer bits with little or no loss of information []. % (as measured by misclassification rate) []. % Consequently, we focus on compressing 8b and 16b integer time series. % Data coming from sensors will almost always be digitized by an Analog-to-Digital Converter (ADC)
\item \b{Lossless}. Given that time series are almost always noisy and often oversampled, it might not seem necessary to compress them losslessly. However, noise and oversampling tend to 1) vary across applications, and 2) be possible to address in an application-specific way as a preprocessing step. Consequently, instead of assuming that some level of downsampling or some particular smoothing will be appropriate for all data, it is better for the compression algorithm to preserve what it is given and leave preprocessing up to the developer.
\end{enumerate}

The primary contribution of this work is \mine (Sprintz PReserves Integer Time Series), a time series compression algorithm that achieves each of these goals. Specifically, it attains state-of-the-art compression ratios and speed, uses $<$1KB of memory, and can use blocks of data as small as eight samples.

Furthermore, thanks to decompression throughput close to memory bandwidth and run-length encoding, \mine can greatly accelerate many queries as compared to other compressors. For a database of autonomous vehicle data, examples of accelerated queries include:

% We introduce a time series compression algorithm that significantly outperforms existing approaches in terms of compression ratio and speed. Because our method can compress blocks as small as eight samples, it is compatible both with data structures that partition time series into small buckets (e.g., []) and low-power devices compressing data at the network edge []. Furthermore, thanks to decompression throughput close to memory bandwidth, our algorithm can greatly accelerate many queries as compared to other compressors. For a database of autonomous vehicle data, examples of accelerated queries include:

\begin{enumerate}[leftmargin=9mm]
\itemsep0mm
\item[\b{Q1}] Compute the maximum speed during each trip
\item[\b{Q2}] Find the top five times that look like the car slipping on ice according to a linear classifier
\item[\b{Q3}] Find all five second intervals containing extreme breaking for a given user
% \item[\b{Q4}] Compute how many hours per week a user spends at work
\end{enumerate}

We also contribute the following:

\begin{itemize}
\itemsep0mm
\item An open-source benchmark suite that allows easy comparison of time series compressors using various combinations of preprocessing, query, and bitwidth.
\item An adaptive predictive coding algorithm with a throughput of up to 6GB/s that can be used indepent of the rest of our method. To the best of our knowledge, this is the fastest single-threaded online learning algorithm ever published (excepting branch predictors and other methods with hardware acceleration).
\end{itemize}

% However, less attention has been paid to the lower-level problem of how to represent the raw values being stored. %, with billions of sensors now streaming data about quantities ranging from acceleration to temperature to heart rate.

% In order to reduce storage costs, this representation should entail some form of compression. However, in order to support the scan-based queries common for time series [], this format must be amenable to extremely fast decompression.

% In order to reduce transmission, power, and storage costs, it is highly desirable to compress this data. For certain types of time series, such as audio \cite{flac, shorten} and timestamps \cite{gorillaDB, fastpfor}, excellent methods already exist. For other time series, this is not the case. One can apply generic compression algorithms, such as GZIP \cite{gzip}, ZSTD \cite{zstd}, LZ4 \cite{lz4}, etc, but as we show, these methods tend to perform poorly on time series. One can also apply integer compression algorithms, such as FastPFOR \cite{fastpfor} and 4Gamma \cite{TODO}, but these suffer from the same drawback.

% Moreover, most existing methods assume computational resources that may be unavailable on low-power devices collecting and transmitting data. In particular, such devices may have less than 1KB of writable memory.

% The contribution of this work is the introduction of \mine (Sprintz PReserves INteger Time SerieZ), a lossless compression algorithm for time series that does not assume a particular application domain (e.g., music) and is suitable for execution on low-power hardware. As we show experimentally, \mine achieves higher compression ratios than any other method across a wide range of datasets while maintaining decompression speeds over 1GB/s in a single thread.

% % ------------------------------------------------
% \subsection{Why lossless?}
% % ------------------------------------------------

% % Given that time series are often represented with too high a sampling rate and bit depth relative to the level of noise

% % and often oversampled or quantized with unneeded precision

% Given that time series are almost always noisy and often oversampled, it might seem unnecessary to compress them losslessly. %---i.e., if the details lost are mostly noise, why bother preserving them?

% However, note that noise and oversampling tend to 1) vary across applications, and 2) be possible to address in an application-specific way as a preprocessing step. Consequently, instead of coupling some particular smoothing, downsampling, or distortion to the compression algorithm and assuming that the result will be ``good enough'' for all data, it is better for the compression algorithm to preserve what it is given and leave preprocessing up to the developer.

% % assuming that some level of downsampling or some particular smoothing will be ``good enough'' for all data, it is better for the compression algorithm to preserve what it is given and leave preprocessing up to the developer.

% % the nature of the noise, oversampling, etc, varies across applications, and 2) it is easy to preprocess the data in an application-specific way to address the before compressing it. Consequently, instead of assuming that some level of downsampling or some particular smoothing will be ``good enough'' for all data, it is better for the compression algorithm to preserve what it is given and leave preprocessing up to the developer.

% % Consequently, it is undesirable to couple denoising to the compression algorithm. Instead, engineers should be free to filter, downsample, quantize, and otherwise condition the data in any way they see fit, and trust that the compression algorithm will preserve the end result. % This applies equally to other operations such as downsampling and reduction of bit depth.

% Moreover, companies cannot necessarily anticipate what sort of degradation might reduce their data's utility for future analysis; this makes haphazardly degrading it through lossy compression risky.

% % ------------------------------------------------
% \subsection{Limited hardware}
% % ------------------------------------------------

% Many connected devices are powered by batteries or harvested energy \cite{bsnChallenges}. This results in strict power budgets and, in order to satisfy them, omission of certain functionality. In particular, many devices lack hardware support for floating point operations, SIMD (vector) instructions, and integer division. Moreover, they often have no more than a few KB of memory, clocks in the tens of MHz at most, and 8-, 16-, or 32-bit processors instead of 64-bit \cite{cc2540, cc2640, quark}.

% We do not assume that the hardware decompressing the data is so limited. Instead, this hardware is likely a modern x86 server with SIMD instructions, gigabytes of RAM, and a multi-GHz clock.

% Given these differing capabilities, a natural question is whether the data must be compressed at the device instead of at the server. Unfortunately, delaying the compression would eliminate much of its benefit. The reason is that reducing the amount of data the device must send offers enormous power savings. To a first approximation, sending data over Bluetooth Low Energy (BLE) costs tens of \textit{milliwatts} while computing at full power costs tens of \textit{microwatts} and sitting idle costs only 1 microwatt \cite{cc2540, cc2640}. Furthermore, compressing the data before the server reduces network load.

% % As a consequence of limited power budgets, there is a strong need to reduce the amount of data transmitted or stored. Indeed, this is the driving motivation for our work. To a first approximation, sending data over Bluetooth Low Energy costs \%d mW and writing to an SD card costs \%d mW; in contrast, computing at full power costs \%d mW and sitting idle costs only \%d mW. This implies that it is worth a great deal of computation to reduce the sizes of writes and transmits.

% % ------------------------------------------------
% \subsection{Nature of the data}
% % ------------------------------------------------

% Almost any time series reflecting a real-world signal will be digitized using an analog-to-digital converter (ADC). This means that the data will be represented as integers of at most 32 bits \cite{digikeyADCs}, and typically 16 or fewer. For example, even lossless audio codecs store only 16 bits \cite{flac, shorten}. Furthermore, there is empirical evidence that most time series can be quantized to 6 or fewer bits with little or no loss of information (as measured by misclassification rate) \cite{epenthesis}. Consequently, we focus on compressing 8b and 16b integer time series. % every ADC currently available through DigiKey, for example, supports 24 or fewer bits as of this writing \cite{digikeyADC}; even high-quality audio is only 16 bits \cite{someAudioCodec}.

% Our method could also be applied to floating point time series insofar as they could be quantized to integers, but we do not evaluate this experimentally since it depends on the quantization method.

% % In addition to being stored as integers, the data is likely to have two additional characteristics. Specifically, it is likely to be 1) constant for long stretches of time and 2) always or sometimes quasi-periodic with unknown period(s). A temperature sensor, for example, will observe nearly identical temperatures for many seconds or minutes, while also experiencing predictable variations over the course of both each day and each year. Similarly, a smart watch will observe long periods of limited movement during sleep or computer use but repetitive movement during walking, running, or swimming. Our method does not require these characteristics to be present, but is designed to exploit them when they are.

% % Finally, we assume that the time series to be compressed are univariate--i.e., they are sequences of scalars. One might be able to obtain better compression by jointly compressing multiple variables known \textit{a priori} to be correlated (e.g. X-axis and Y-axis acceleration), but we do not do this since:
% % \begin{enumerate}[leftmargin=4mm]
% % \item It results in read amplification if only one variable is of interest.
% % \item It tends not to increase compression very much.
% % \item It complicates the algorithm, since inter-variable correlations can switch unpredictably between being positive and negative.
% % \item It makes direct comparison to existing work more difficult.
% % \end{enumerate}

% % ------------------------------------------------
% % \subsection{Limited hardware}
% % ------------------------------------------------





% % ================================================================
% \section{Definitions and Problem} \label{sec:problem}
% % ================================================================

% \input{problem.tex}

% % ================================================================
% \section{Related Work} \label{sec:relatedWork}
% % ================================================================

% 
% ------------------------------------------------
\subsection{Compression of time series}
% ------------------------------------------------

Most work on compressing time series has focused on lossy techniques. In the data mining literature, Symbolic Aggregate Approximation (SAX) \cite{sax} and its variations \cite{isax, isax2} dominate. These approaches preserve enough information about time series to support specific data mining algorithms (e.g. \cite{fastShapelet, hotSax}), but are extremely lossy; a hundred-sample time series might be compressed into one or two bytes, depending on the exact SAX parameters.

Other lossy approaches include Adaptive Piecewise Constant Approximation \cite{apca}, Piecewise Aggregate Approximation \cite{paa}, and numerous other methods \cite{swab, lemireSegmentation, tsCompressSmartGrid} that approximate time series as sequences of low-order polynomials.

For audio time series specifically, there are a large number of lossy codecs \cite{vorbis, shorten, aac, opus}, as well as a small number of lossless \cite{flac, alac} codecs. In principle, some of these could be applied to non-audio time series. However, modern codecs make such strong assumptions about the possible numbers of channels, sampling rates, bit depths, or other characteristics that it is infeasible to use them on non-audio time series.

Many fewer algorithms exist for lossless time series compression. For floating-point time series, the only algorithm of which we are aware is that of the Gorilla database \cite{gorilla}. This method XORs each value with the previous value to obtain a diff, and then bit packs the diffs. In contrast to our approach, it assumes that time series are univariate and have 64-bit floating-point elements. % The same work also describes a method of compressing integer timestamps. This consists of first delta-delta coding and then applying a similar bit packing compression approach.
% Like other databases \cite{influxDB, berkeleyTreeDB}, they delta-delta code before applying compression %Most time series databases use some form of integer compression (c.f. next section) [], generic predictive coding \cite{akumuli}, or floating-point compression methods.

For lossless compression of integer time series (including timestamps), existing approaches include directly applying general-purpose compressors \cite{respawnDB, openTSDB, chronicleDB, kairosDB, druid}, (double) delta encoding and then applying an integer compressor \cite{influxDB, gorilla}, or predictive coding and byte-packing \cite{akumuli}. These approaches can work well, but tend to offer both less compression and less speed than \mine.

% A final noteworthy method is Blosc \cite{blosc}. While not intended solely for time series (or integers), Blosc's assumption that every $k$ bytes (or bits \cite{bitshuf}) will be correlated for some $k$ is a natural fit for multivariate time series. %grouping of correlated bytes and/or bits makes it well-suited for multivariate time series.


% ------------------------ parquet
%   -currently supports snappy, GZIP, lzo
%   -hdfs also supports these; prolly others also
%   -https://www.cloudera.com/documentation/enterprise/5-6-x/topics/impala_parquet.html#parquet_compression
% ------------------------ Akumuli
%   -DFCM predictor (for both floats and ints?); XOR with prediction, then do something to pack it
%   -"The timestamp can be a simple integer or datetime in ISO 8601 format"
%   -handles int or float values
%   -"This data-structure consists of 4KB blocks"
%   -"They require large amount of memory per data stream to maintain a sliding window of previously seen samples. The larger the context size the better compression ratio gets."
%   -"if we're dealing with 100'000 time-series we'll need about 1GB of memory only for compression contexts."
% ------------------------ RespawnDB
%   -GZIP
%   -"Enabling compression adds a factor of six slowdown in BTDS performance"
% ------------------------ OpenTSDB
%   -LZ0 or snappy for floats, varbyte (1, 2, 4, or 8 bytes) for ints
%   -http://opentsdb.net/faq.html
% ------------------------ Gorilla
%    -custom F64 for values, custom delta-delta for timestamps
% ------------------------ ChronicleDB
%   -lz4
% ------------------------ BerkeleyTreeDB
%   -custom delta-delta
% ------------------------ KairosDB
%     -off-the-shelf compressors: lzo, snappy, probably others
%     -https://github.com/kairosdb/kairosdb/search?utf8=✓&q=compression&type=
% ------------------------ Druid
%   -"The timestamp and metric columns are simple: behind the scenes each of these is an array of integer or floating point values compressed with LZ4."
%       -"metrics" are scalars (the time series)
%   -they also have "dimensions" which are categorical cols; assign each a numeric ID and map IDs to lists of places they occur
%   -http://druid.io/docs/latest/design/segments.html
% ------------------------ InfluxDB
%   -"Timestamp encoding is adaptive and based on the structure of the timestamps that are encoded. It uses a combination of delta encoding, scaling, and compression using simple8b run-length encoding"
%   -"Floats are encoded using an implementation of the Facebook Gorilla paper."
%   -For integers: "If all ZigZag encoded values are less than (1 << 60) - 1, they are compressed using simple8b encoding. If any values are larger than the maximum then all values are stored uncompressed in the block. If all values are identical, run-length encoding is used."
%   -"Strings are encoding using Snappy compression"
%   -and before compaction (where all the above methods get used): "When a write comes in the new points are serialized, compressed using Snappy, and written to a WAL file"
%       -and note: "This means that batching points together is required to achieve high throughput performance. (Optimal batch size seems to be 5,000-10,000 points per batch for many use cases.)"
% ------------------------ LittleTable
%   -LZO1X (LZO) compression http://www.oberhumer.com/opensource/lzo/
% ------------------------ RocksDB (not a ts database, but whatever)
%   -snappy, lz4, zstd


\subsection{Compression of integers}

% Integer compression is not as well-studied as general-purpose compression, but has seen great progress in recent years.

The fastest methods of compressing integers are generally based on bit packing---i.e., using at most $b$ bits to represent values in $\{0, 2^b-1\}$, and storing these bits contiguously \cite{bbp, pfor, fastpfor}. Since $b$ is determined by the largest value that must be encoded, naively applying this method yields limited compression. To improve it, one can encode fixed-size blocks of data at a time, so that $b$ can be set based on the largest values in a block instead of the whole dataset \cite{kGamma, pfor, fastpfor}. A further improvement is to ignore the largest few values when setting $b$ and store their omitted bits separately \cite{pfor, fastpfor}.

\minesp bit packing differs significantly from existing methods in two ways. First, it compresses much smaller blocks of samples. This reduces its throughput as compared to, e.g., \cite{fastpfor}, but significantly improves compression ratios (c.f. Section~\ref{sec:results}). This is because large values only increase $b$ for a few samples instead of for many. Second, \minesp is designed for 8 and 16-bit integers, rather than 32 or 64-bit integers. Existing methods are often inapplicable to lower-bitwidth data (unless converted to higher-bitwidth data) thanks to strong assumptions about bitwidth and data layout.
% $b$ that is even one too large significantly alters the ratio for low-bitwidth data.

A common \cite{flac, shorten} alternative to bit packing is Golomb coding \cite{golomb}, or its special case Rice coding \cite{rice}. The idea is to assume that the values follow a geometric distribution, often with a rate constant fit to the data. %, and therefore make the encoding cost linear in the magnitude of the encoded value.

Both bit packing and Golomb coding are bit-based methods in that they do not guarantee that encoded values will be aligned on byte boundaries. When this is undesirable, one can employ byte-based methods such as 4-Wise Null Suppression \cite{kGamma}, LEB128 \cite{dwarf}, or Varint-G8IU \cite{varintG8IU}. These methods reduce the number of bytes used to store each sample by encoding in a few bits how many bytes are necessary to represent its value, and then encoding only that many bytes. Some, such as Simple8B \cite{simple8b} and SIMD-GroupSimple \cite{groupSimd}, allow fractional bytes to be stored while preserving byte alignment for groups of samples. % These methods allow for efficient universal codes---that is, codes that can represent any possible integer. Universal

% Before applying any of these coding schemes, it is common to apply some transform to the raw data to make the values closer to 0. Such transforms include delta encoding, \cite{fastpfor, bbp}, delta-delta encoding \cite{influxDB}, and linear predictive coding (LPC) \cite{flac}. LPC deterministically generates a prediction for each sample based on the previous samples, and stores the error in the prediction instead of the raw value; when the errors are small, the integers stored are closer to 0. Delta coding and delta-delta coding are special cases wherein each sample is predicted to be the previous sample, or a linear extrapolation from the previous two samples, respectively.

\subsection{General-purpose compression}
While \minesp is not intended to be a general-purpose compression algorithm, a reasonable alternative to using a specialized method would be to apply a general-purpose compression algorithm, possibly after delta coding or other preprocessing. Thanks largely to the development of Asymmetric Numeral Systems (ANS) \cite{ans} for entropy coding, general purpose compressors have advanced greatly in recent years. In particular, Zstd \cite{zstd}, Brotli \cite{brotli}, LZ4 \cite{lz4} and others have attained speed-compression tradeoffs significantly better than traditional methods such as GZIP \cite{gzip}, LZO \cite{lzo}, etc. As described in Section~\ref{sec:intro}, however, these methods have much higher memory requirements that \mine.

% Also of note is Blosc \cite{blosc}, which is especially applicable to multivariate time series as a result of its grouping correlated bits and/or bytes together during preprocessing.

% ------------------------------------------------
\subsection{Predictive Filtering}
% ------------------------------------------------

% TODO move delta, double, LPC descriptions to here. Also talk about \fire.

% Predictive coding in some form is a common element of many compression algorithms. This consists of having some forecaster predict the values of the next byte(s) to be compressed and only storing the prediction error, rather than the original value. When the forecaster is better than chance, the errors will be drawn from a lower-entropy distribution than that of the raw data. In particular, they will often be tightly concentrated around zero \cite{shorten}. The original data can be reconstructed from the errors by having the decoder run the same forecaster and add the encoded errors to its predictions.

For numeric data such as time series, there are four types of predictive coding commonly in use: predictive filtering \cite{png}, delta coding \cite{fastpfor, bbp}, double-delta coding \cite{influxDB, gorilla}, and XOR-based encoding \cite{gorilla}. In predictive filtering, each prediction is a linear combination of a fixed number of recent samples. This can be understood as an autoregressive model or the application of a Finite Impulse Response (FIR) filter. When the filter is learned online, this is termed ``adaptive filtering.''

Delta coding is a special case of predictive filtering where the prediction is always the previous value. Double-delta coding, also called delta-delta coding or delta-of-deltas coding, consists of applying delta coding twice in succession. XOR-based encoding is similar to delta coding, but replaces subtraction of the previous value with the XOR operation. This modification is often desirable for floating-point data \cite{gorilla}.

\fire can be understood as a special case of adaptive filtering. While adaptive filtering is a well-studied mathematical problem in the signal processing literature, we are unaware of a practical algorithm that attains speed within an order of magnitude of that of \justfire. %. or that accounts for the subtleties of low-precision integers.

% Indeed, the only methods

% most similar method is likely that of \cite{neuralBranchPredictor}, in which a

% Moreover, existing databases almost universally use delta, double-delta, or XOR encoding, suggesting that simple  % suggesting that these approaches are the relevant benchmarks.

% Before applying any of these coding schemes, it is common to apply some transform to the raw data to make the values closer to 0. Such transforms include delta encoding, \cite{fastpfor, bbp}, delta-delta encoding \cite{influxDB}, and linear predictive coding (LPC) \cite{flac}. LPC deterministically generates a prediction for each sample based on the previous samples, and stores the error in the prediction instead of the raw value; when the errors are small, the integers stored are closer to 0. Delta coding and delta-delta coding are special cases wherein each sample is predicted to be the previous sample, or a linear extrapolation from the previous two samples, respectively.


% ================================================================
\section{Method} \label{sec:method}
% ================================================================


We totally have a method. And golly, does it ever method.


% ================================================================
\section{Experimental Results} \label{sec:results}
% ================================================================


% To assess \mine's effectiveness, we implemented both it and comparison algorithms in C++.

To asses \mine's effectiveness, we compared it to a number of state-of-the art compression algorithms on a large set of publicly available datasets. All of our code and raw results are publicly available on the \minesp website.\footnote{https://smarturl.it/sprintz} This website also contains additional experiments, as well as documentation of both our code and experimental setups. All experiments use a single thread on a 2013 Macbook Pro with a 2.6GHz Intel Core i7-4960HQ processor.

All reported timings and throughputs are the best of ten runs. We use the best, rather than average, since this is 1) desirable in the presence of the non-random, purely additive noise characteristic of microbenchmarks, and, 2) consequently, a best practice in microbenchmarking \cite{lemireMicrobenchmarks}. The best values are nearly always within 10\% of the averages.

% ------------------------------------------------
\subsection{Datasets}
% ------------------------------------------------

% For assessing accuracy, we use several publicly available datasets:
\begin{itemize}[leftmargin=4mm]
\itemsep0mm
\item \textbf{UCR} \cite{ucrTimeSeries} --- The UCR Time Series Archive is a repository of 85 univariate time series datasets from various domains, commonly used for benchmarking time series algorithms. Because each dataset consists of many (often short) time series, we concatenate all the time series from each dataset to form a single longer time series. This is to allow dictionary-based methods to share information across time series (instead of compressing each in isolation). To mitigate artificial jumps in value from the end of one time series to the start of another, we linearly interpolate five samples between each pair.
% Because the file format is delimited text with labels interspersed with data, we extract ``raw'' data by reading the time series within each dataset into a contiguous array of the appropriate data type We concatenated the first 100 examples from each of the 85 time series datasets in the UCR Time Series Archive \cite{ucrTimeSeries} to form 85 longer time series. Before concatenating, we subtracted off the mean from each example and interpolated one sample between its end and the start of the next example to avoid sudden jumps. This processing makes the datasets in some sense synthetic, but the result is an easy-to-reproduce benchmark incorporating time series from dozens of domains. We report aggregate statistics across these datasets.
\item \textbf{PAMAP} \cite{pamap} --- The PAMAP dataset consists of inertial motion and heart rate data from wearable sensors on subjects performing everyday actions. It has 31 variables, most of which are accelerometer and gyroscope readings.
\item \textbf{MSRC-12} \cite{msrc} --- The MSRC-12 dataset consists of 80 variables of (x, y, z, depth) positions of human joints captured by a Microsoft Kinect. The subjects performed various gestures one might perform when interacting with a video game.
% \item \textbf{WARD} \cite{ward} --- Berkeley Ward Dataset; maybe use this
% \item \textbf{ECG} \cite{physiobank} --- Some big ECG dataset from physiobank.
\item \textbf{UCI Gas} \cite{uci_gas} --- This dataset consists of 18 columns of gas concentration readings and ground truth concentrations during a chemical experiment.
\item \textbf{AMPDs} \cite{ampds} --- The Almanac of Minutely Power Datasets describes electricity, water, and natural gas consumption recorded once per minute for two years at a single home. % We treat the data from each of these modalities as one dataset and report aggregate performance across all three.
\end{itemize}

% TODO maybe split ampds into 3 different datasets and report on each separately.

For datasets stored as delimited files, we first parsed the data into a contiguous, numeric array and then dumped the bytes as a binary file. Before obtaining any timing results, we first load each dataset into main memory.
% All datasets are represented as contiguous numeric arrays of the appropriate integral datatype for the bitwidth and loaded in main memory.
Because the datasets are provided as floating point values (despite most reflecting analog-to-digital converter output that was originally integer-valued), we quantized them into integers before operating on them. We did so by linearly rescaling them such that the largest and smallest values corresponded to the largest and smallest values representable with the number of bits tested---e.g., 0 and 255 for 8 bits---and then applying the floor function. Note that this is the worst case scenario for our method since it maximizes the number of bits required to represent the data.

For multivariate datasets, we allowed all methods but our own to operate on the data one variable at a time; i.e., instead of interleaving values for every variable, we store all values for each variable contiguously. This corresponds to allowing them an unlimited buffer size in which to store incoming data before compressing it. We allow these ideal conditions in order to ensure that our results for existing methods err towards optimism and to eliminate buffer size as a lurking variable.

% For multivariate datasets, we concatenated the data from each variable to obtain a univariate time series. As discussed previously, one might be able to obtain better performance by jointly compressing the variables, but we defer this to future work since it both makes direct comparison to existing methods more difficult and complicates the algorithm.

% ------------------------------------------------
\subsection{Comparison Algorithms}
% ------------------------------------------------

\begin{itemize}[leftmargin=4mm]
\itemsep0mm
\item \textbf{SIMD-BP128} \cite{fastpfor} --- The fastest known method of compressing integers.
\item \textbf{FastPFOR} \cite{fastpfor} --- An algorithm similar to SIMD-BP128, but with better compression ratios.
\item \textbf{Simple8b} \cite{simple8b} --- An integer compression algorithm used by the popular time series database InfluxDB \cite{influxDB}.
% \item \textbf{LZO} \cite{lzo} --- A stable and widely-used dictionary compressor employed by KairosDB \cite{kairosDB} and LittleTable \cite{littleTable}.
\item \textbf{Snappy} \cite{snappy} --- A general-purpose compression algorithm developed by Google and used by InfluxDB, KairosDB \cite{kairosDB}, OpenTSDB \cite{openTSDB}, RocksDB \cite{rocksDB}, the Hadoop Distributed File System \cite{hdfs} and numerous other projects.
\item \textbf{Zstd} \cite{zstd} --- Facebook's state-of-the-art general purpose compression algorithm. It is based on LZ77 and entropy codes using a mix of Huffman coding and Finite State Entropy (FSE) \cite{fse}. It is available in RocksDB \cite{rocksDB}.
% \item \textbf{Brotli} \cite{brotli} --- A recent compression algorithm introduced by Google and now standardized as a web content-encoding type.
\item \textbf{LZ4} \cite{lz4} --- A widely-used general-purpose compression algorithm optimized for speed and based on LZ77. It is used by RocksDB and ChronicleDB \cite{chronicleDB}.
% \item \textbf{LZ4-HC} \cite{lz4} --- A variant of LZ4 optimized for compression ratio at the cost of compression speed.
\item \textbf{Zlib} \cite{zlib} --- A popular implementation of the DEFLATE \cite{deflate} dictionary coder, which also underlies gzip \cite{gzip}.

% The compression algorithm underlying zlib \cite{zlib} and gzip \cite{gzip}. Used by RespawnDB \cite{respawnDB}, the Parquet columnar storage format \cite{parquet}, and HDFS \cite{hdfs}. We use the zlib implementation.
% \item \textbf{BitShuf} \cite{bitshuf} --- LZ4 with the BitShuffle preprocessor, which groups runs of 0 bits when consecutive values are similar and small.
% \item \textbf{Delta+BitShuf} \cite{gzip} --- Like BitShuf, but applied to the delta-encoded representation of the time series.
% \item \textbf{Delta} --- Simple delta encoding followed by...erm...some kind of bit packing or something.
% \item \textbf{DeltaDelta} --- Delta encoding the delta encoding, as done in some popular time series databases \cite{something, influxDB}.
\end{itemize}

For Zlib and Zstd, we use a compression level of 9 unless stated otherwise. This level heavily prioritizes compression ratio at the expense of increased compression time. We use it to improve the results for these methods in experiments in which compression time is not penalized.

We also assess three variations of \mine, corresponding to different speed/ratio tradeoffs:% Following the convention used by existing compressors, we identify these variations as compression levels 1 to 3:
% \begin{enumerate}[leftmargin=2mm]
\begin{enumerate}
% \itemsep-1mm
    \item \textbf{\texttt{SprintzFIRE+Huf}}. The full algorithm described in Section~\ref{sec:method}.
    \item \textbf{\texttt{SprintzFIRE}}. Like \texttt{SprintzFIRE+Huf}, but without Huffman coding.
    \item \textbf{\texttt{SprintzDelta}}. Like \texttt{SprintzFIRE}, but with delta coding instead of \fire as the forecaster.
\end{enumerate}

% We also assess the above methods when applied to the delta-encoded representation of the time series, as well as the double-delta-encoded representation. We do not do this for FLAC and FastPFOR since they have similar preprocessing steps built in.

% Note that all of the above except FastPFOR, and possibly FLAC with special configuration, require tens of KB, or even tens of MB, of memory, and therefore are unsuitable for many low-power devices.

% ------------------------------------------------
\subsection{Compression Ratio}
% ------------------------------------------------

In order to rigorously assess the compression performance of both \minesp and existing algorithms, it is desirable to evaluate each on a large corpus of time series from heterogeneous domains. Consequently, we use the UCR Time Series Archive \cite{ucrTimeSeries}. This corpus contains dozens of datasets and is almost universally used for evaluating time series classification and clustering algorithms in the data mining community.

The distributions of compression ratios on these datasets for the above algorithms are shown in in Figure~\ref{fig:ratioBox}. \minesp exhibits consistently strong performance across almost all datasets. High-speed codecs such as Snappy, LZ4, and the integer codecs (FastPFOR, SIMDBP128, Simple8B) hardly compress most datasets at all.

\begin{figure}[h]
\begin{center}
    \includegraphics[width=10cm]{paper/boxplot_ucr}
    \caption{Boxplots of compression performance of different algorithms on the UCR Time Series Archive. Each boxplot captures the distribution of one algorithm across all 85 datasets.}
    \label{fig:ratioBox}
\end{center}
\end{figure}

Perhaps counter-intuitively, 8-bit data tends to yield higher compression ratios than 16-bit data. This is a product of the fact that the number of bits that are ``predictable'' is roughly constant. I.e., suppose that an algorithm can correctly predict the four most significant bits of a given value; this enables a 2:1 compression ratio in the 8-bit case, but only a 16:12 = 4:3 ratio in the 16-bit case. Interestingly, the fact that trailing bits tend to be too noisy to compress also suggests that one could use a lower bitwidth with little loss of information. % A more concise explanation is that all but the top few bits are difficult to distinguish from noise, so the larger the fraction of the value these bits constitute, the greater the compressibility.

To assess \mine's performance statistically, we use a Nemenyi test \cite{nemenyiTest} as recommended in \cite{cdDiagrams}. This test compares the mean rank of each algorithm across all datasets, where the highest-ratio algorithm is given rank 1, the second-highest rank 2, and so on. The intuition for why this test is desirable is that it not only accounts for multiple hypothesis testing in making pairwise comparisons, but also prevents a small number of large or highly compressible datasets from dominating the results.
%To see why the latter attribute is important, consider two obvious alternatives to using ranks. If we treated the entire archive as one blob and reported the overall ratio, the metric would be dominated by performance on a small number of especially large datasets. If we instead compressed each dataset separately but operated on the ratios, instead of ranks, datasets with high ratios (and likely high variance) would dominate. By using ranks on each dataset, the Nemenyi test avoids both of these pitfalls.

 % E.g., if we were to compress the entire archive as one blob, the largest several datasets would dominate the metric. Similarly, if we averaged the ratios, the datasets with highest variance would dominate. % admiting the largest and/or higest-variance ratios would dominate. % admiting the largest ratios would dominate.

The results of the Nemenyi test are shown in the Critical Difference Diagrams \cite{cdDiagrams} in Figure~\ref{fig:ratioCD}. These diagrams show the mean rank of each algorithm on the x-axis and join methods that are not statistically significantly different with a horizontal line. \minesp on high compression settings is significantly better than any existing algorithm. On lower settings, it is still as effective as the best current methods (Zlib and Zstd).

%  it is important to choose an appropriate overall metric. One obvious choice would be to simply measure the total size of all compressed datasets compared to the original size. Unfortunately, this would cause a small number of large datasets to dominate. Even if we took a fixed number of time series per dataset, datasets with longer time series would still contribute most of the data.

% Another option would be to compute the compression ratio for each dataset and average these numbers. This is also undesirable for similar reasons. Specifically, it allows performance on a small number of highly compressible datasets to dominate the overall metric.

% Because these considerations closely parallel those associated with comparing classifiers across multiple datasets, we use the methodology outlined in \cite{_Diagrams}. This means computing the rank of each algorithm for each dataset and comparing the mean ranks using a Nemenyi test. A rank of 1 indicates the best ratio, while 2 indicates the second-best ratio, and so on.

% Results using this methodology are shown as a Critical Difference Diagram \cite{cdDiagrams} in Figure~\ref{fig:ratioCD}. Sprintz on high compression settings is significantly better than any existing algorithm. On slightly lower settings, it is still as effective as the best current methods (Zlib and Zstd).

\begin{figure}[h]
\begin{center}
    \includegraphics[width=10cm]{paper/cd_diagram_8b}
    \includegraphics[width=10cm]{paper/cd_diagram_16b}
    \caption{Compression performance of different algorithms on the UCR Time Series Archive. The x-axis is the mean rank of each method, where rank 1 on a given dataset had the highest ratio. Methods joined with a horizontal black line are not statistically significantly different.}
    \label{fig:ratioCD}
    \vspace{-5mm}
\end{center}
\end{figure}

In addition to this overall comparison, it is important to assess whether \fire improves performance compared to delta coding. Since this is a single hypothesis with matched pairs, we assess it using a Wilcoxon signed rank test. This yields p-values of .0094 in the 8-bit case and 4.09e-12 in the 16-bit case. As a more interpretable measure, \fire obtains better compression on 51 of 85 datasets using 8 bits and 74 of 85 using 16. These results suggest that \fire is generally beneficial on 8-bit data but even more beneficial on 16-bit data.

To understand why 16-bit data benefits more, we examined datasets where \fire gives differing benefits in the two cases. The difference most commonly occurs when the data is highly compressible with just delta coding. With 8 bits and $\sim$$4\times$ compression, the forecaster's task is effectively to guess whether the next delta is -1, 0, or 1 given a current delta drawn from this same set. The Bayes error rate is high for this problem, and \justfire's attempt to learn adds variance compared to the delta coding approach of always predicting 0. In contrast, with 16 bits, the deltas span many more values and retain continuity that \fire can exploit.

% As a more intuitive illustration, we also include the distributions of raw compression ratios (Figure~\ref{fig:ratioBox}). \minesp exhibits consistently strong performance across almost all datasets. High-speed codecs such as Snappy, LZ4, and the integer codecs (FastPFOR, SIMDBP128, Simple8B) hardely compress most datasets at all.


% Using the 85 UCR datasets:
% \begin{enumerate}
% \item CD Diagram of \mine (all levels) vs other algorithms
% \item CD Diagram of \mine (all levels) vs delta coding + other algorithms
% \item If we don't win in at least the first one, CD Diagram of \mine (all levels) vs delta coding + other algorithms that use very little memory
% \end{enumerate}

% % ------------------------------------------------
% \subsection{Ratio-Speed Tradeoff}
% % ------------------------------------------------

% A natural question is whether the above compression ratios come at the price of reduced speed. To test this, we recorded the speeds and compression ratios of both our method and others on several multivariate datasets. We do not simply reuse the UCR datasets because they are all univariate, which is both not our algorithm's focus and its worst case.


% ------------------------------------------------
\subsection{Decompression Speed} \label{sec:decomp_speed}
% ------------------------------------------------

To systematically assess the speed of \minesp, we ran it on time series with varying numbers of columns and varying levels of compressibility. Because real datasets have a fixed and limited number of columns, we ran this experiment on synthetic data. Specifically, we generated a synthetic dataset of 100 million random values uniformly distributed across the full range of those possible for the given bitwidth. This data is incompressible and thus provides a worst-case estimate of \mine's speed (though in practice, we find that the speed is largely consistent across levels of compressibility).% Even with Huffman coding, we have not observed more than a 30\% speedup from greater compressibility.

We compressed the data with \minesp set to treat it as if it had 1 through 80 columns. Numbers that do not evenly divide the data size result in \minesp \texttt{memcpy}-ing the trailing bytes.

While using this synthetic data cannot tell us anything about \mine's compression ratio, it is suitable for throughput measurement. This is because both \mine's sequence of instructions executed and memory access patterns are effectively independent of the data distribution---\mine's core loop has no conditional branches and \minesp's memory accesses are always sequential. Moreover, it exhibits throughputs on real data matching or slightly exceeding the numbers below for the corresponding number of columns (c.f. Figure~\ref{fig:tradeoff_success}). % thanks to the distribution-independence of \mine's execution time. I.e., \minesp has no conditional branches in its core loop and always reads and writes sequentially. Moreover, it exhibits throughputs on real data closely matching the numbers below for the corresponding number of columns (c.f. Figure~\ref{fig:tradeoff_success}).

As shown in Figure~\ref{fig:ndims_vs_decomp_speed}, \minesp becomes faster as the number of columns increases and as the number of columns approaches multiples of 32 for 8-bit data or 16 for 16-bit data. These values correspond to the 256-bit width of a SIMD register on the machine used for testing. There is small but consistent overhead associated with using \fire over delta coding, but both approaches are extremely fast. Without Huffman coding, \minesp decompresses at multiple GB/s once rows exceed $\sim$16B. With Huffman coding, the other components of \minesp are no longer the bottleneck and \minesp consistently decompresses at over 500MB/s. Note that we omit comparison to other algorithms in this section since their speed varies with compressibility, not number of columns; see Section~\ref{sec:whenSprintz} for a direct comparison. Further note that the speed's dependence on number of columns is not an artifact of more columns yielding larger blocks of data. The limiting factor is serial dependence between decoding one sample and predicting the next one; this is accelerated by having wider samples that fill a vector register, but not by having longer blocks.

\begin{figure}[h]
\begin{center}
    \includegraphics[width=10cm]{paper/ndims_vs_decomp_speed}
    \caption{\minesp becomes faster as the number of columns increases and as the width of each sample approaches multiples of 32B (on a machine with 32B vector registers). }
    % Its speed is robust to the compression ratio, as illustrated by the speeds in the top row (low compression) roughly equaling those in the bottom row (high compression).
    \label{fig:ndims_vs_decomp_speed}
\end{center}
\end{figure}


% ------------------------------------------------
\subsection{Compression Speed} \label{sec:comp_speed}
% ------------------------------------------------

It is important that \mine's compression speed be fast enough to keep up with the rate of data ingestion. We measured \mine's compression speed using the same methodology as decompression speed. As shown in Figure~\ref{fig:ndims_vs_comp_speed}, \mine \text{} compresses 8-bit data at over 200MB/s on the highest-ratio setting and 600MB/s on the fastest setting. These numbers are roughly 50\% larger on 16-bit data. We refrained from vectorizing this prototype implementation because 1) 200MB/s is already fast enough to run in real time even if every thread were fed data from its own gigabit network connection, and 2) low-power devices often lack vector instructions, so the measured speeds are more indicative of the rate at which these devices could compress (if scaled to the appropriate clock frequency). We again omit comparison to other compressors for the same reason as in the previous section.

\begin{figure}[h]
\begin{center}
    \includegraphics[width=10cm]{paper/ndims_vs_comp_speed}
    \caption{\minesp compresses at hundreds of MB/s even in the slowest case: its highest-ratio setting with incompressible 8-bit data. On lower settings with 16-bit data, it can exceed 1GB/s.}
    \label{fig:ndims_vs_comp_speed}
\end{center}
\end{figure}

% Same thing for compression speed (Figure~\ref{fig:ndims_vs_comp_speed}). Except we didn't vectorize the compressor so it's pretty insensitive to dimensionality. We refrained from doing so since 1) the resulting implementation would exactly mirror the decompressor, and have similar performance characteristics, so maybe we can justify not spending the effort; and 2) because this is more analogous to what one would expect if the compressor were run on a low-power device, which would almost certainly lack vector instructions. We want to argue that it only needs to be fast enough to keep up with data rate.

The dips after 4 columns in 8-bit data and 2 columns in 16-bit data correspond to the switch from column-major bit packing to rowmajor bit packing.

% ------------------------------------------------
\subsection{FIRE Speed}
% ------------------------------------------------

To characterize the speed of the \fire we repeated the above throughput experiments with both it and two other predictors commonly seen in the literature: delta and double delta coding. As shown in Figure~\ref{fig:ndims_vs_preproc_speed}, \fire can encode at up to 5GB/s and decode at up to 6GB/s. This is nearly the same speed as the competing methods and close to the 7.5 GB/s speed of \texttt{memcpy} on the tested machine. Note that ``encode'' and ``decode'' here mean converting raw samples to errors and reconstructing samples from sequences of errors, respectively. These operations do not change the data size, but are the subroutines run in the \minesp compressor and decompressor. The reason that there is less discrepancy between delta and \fire encoding in isolation versus when embedded in \minesp compression (Figure~\ref{fig:ndims_vs_comp_speed}) is that, in this experiment, the implementations are vectorized.

\begin{figure}[h]
\begin{center}
    \includegraphics[width=10cm]{paper/ndims_vs_preproc_speed}
    \caption{\fire is nearly as fast as delta and double delta coding. For a moderate number of columns, it runs at 5-6GB/s on a machine with 7.5GB/s \texttt{memcpy} speed.}
    \label{fig:ndims_vs_preproc_speed}
\end{center}
\end{figure}

% ------------------------------------------------
\subsection{When to Use Sprintz} \label{sec:whenSprintz}
% ------------------------------------------------

% The above speed benchmarks provide a fairly

The above experiments provide a characterization of \mine's speed and a statistically meaningful assessment of its compression ratio in general. However, because one often wants to obtain the best results on a particular type of data, it is helpful to know when \minesp is likely to work well or poorly.

Regarding speed, \minesp is most desirable when there are many variables to be compressed. We have found that the speed is largely insensitive to compression ratio, so the results in Sections~\ref{sec:decomp_speed}~and~\ref{sec:comp_speed} offer a good estimate of the speed one could expect on similar hardware. The exception to this is if the data contains long runs of constants (or constant slopes if using \justfire). In this case, the decompression speed approaches the speed of \texttt{memcpy} for \texttt{SprintzDelta} or the speed of \fire for \texttt{SprintzFIRE} and \texttt{SprintzFIRE+Huf}.

% we have found that the dimensionality and bitwidth almost completely determine \mine's throughput (c.f. Sections~\ref{sec:decomp_speed},~\ref{sec:comp_speed}). The exception is when there are long runs of constants, in which case the decompression speed approaches the speed of \texttt{memcpy} for \texttt{SprintzDelta} or the speed of \fire for \texttt{SprintzFIRE} and \texttt{SprintzFIRE+Huf}.
% TODO run RLE stuff for speed vs ndims and then ref the sections at end of this paragraphor \

% This relationship holds almost any time one chooses to measure the quantity with reasonable fidelity.

% Regarding accuracy, there are several relevant data characteristics.
Regarding compression ratio, the dominant requirement is that the data must have relatively strong correlations between consecutive values. This occurs when the sampling rate is fast relative to the time scale over which the measured quantity changes---the typical case when one seeks reasonably high-quality measurements. When these correlations are absent, predictive filtering (with only a two-component filter) has little value. Indeed, it can even be counterproductive. Consider the case of data that has an isolated nonzero value every few samples---e.g., the sequence $\{0, -1, 0, 0\}$. When delta coded, this yields $\{0, -1, 1, 0\}$, which requires an extra bit for \minesp bit packing. In general, \minesp has to pay the cost of abrupt changes twice---once when they happen, and once when they ``revert'' to the previous level.

\begin{figure}[t]
\begin{center}
    \includegraphics[width=10cm]{paper/tradeoff_success}
    \caption{\minesp achieves excellent compression ratios and speed on relatively slow-changing time series with many variables.}
    \label{fig:tradeoff_success}
\end{center}
    \vspace{-5mm}
\end{figure}

Another specific case in which \minesp is undesirable is when the data distribution tends to switch between discrete states. For example, in electricity consumption data, an appliance tends to use little or no electricity when it is off and a relatively constant amount when it is on. Switches between these states are expensive for \mine, and predictive filtering offers little benefit on sequences of samples that are already almost constant. \minesp can still achieve reasonably good compression in this situation, but dictionary-based compressors will likely perform better. This is because they suffer no penalty from state changes, and runs of constants are their best-case input in terms of both ratio and speed. Their ratio benefits because they can often run-length encode the number of repeated values, and their speed benefits because they can decode runs at memory speed by \texttt{memcpy}-ing the repeated values.

As an illustration of when \minesp is and is not preferable, we ran it and the comparison algorithms on several real-world datasets with differing characteristics. In Figure~\ref{fig:tradeoff_success}, we use the MSRC-12, PAMAP and UCI Gas datasets. These datasets contain time series that change slowly relative to the sampling rate and have 80, 31, and 18 variables, respectively. \minesp achieves an excellent ratio-speed tradeoff on all three datasets, and the highest compression of any method \textit{even on its lowest-compression setting} on the MSRC-12 dataset.

In contrast, \minesp performs poorly on the AMPD Gas and AMPD Water datasets (Figure~\ref{fig:tradeoff_failure}). These datasets chronicle the natural gas and water consumption of a house over a year, and often switch between discrete states and/or have isolated nonzero values. They also have only three and two variables, respectively. \minesp achieves more than $10\times$ compression, but dictionary-based methods such as Zstd and LZ4 achieve even greater compression, while also decompressing faster.

\begin{figure}[h]
\begin{center}
    \includegraphics[width=10cm]{paper/tradeoff_fail}
    \caption{\minesp is less effective than other methods when the time series has large, abrupt changes and few variables.}
    \label{fig:tradeoff_failure}
\end{center}
\end{figure}

% ------------------------------------------------
\subsection{Generalizing to Floats} \label{sec:floats}
% ------------------------------------------------

While floating-point values are not the focus of this work, it is possible to apply \minesp to floats by first quantizing the floating-point data. The downside of doing this is that, because floating-point numbers are not uniformly distributed along the real line, such quantization is lossy. To assess the degree of loss, we carried out an experiment to measure the error induced when quantizing real data. Note that this experiment does not assess whether \minesp is the \textit{best} means of compressing floats---it merely suggests that using integer compressors like \minesp as lossy floating-point compressors is reasonable and could be a fruitful avenue for future work. % We leave rigorous assessment of whether quantization followed by \minesp is the \textit{best} means of compressing floating-point data to future work.

We assessed the magnitude of typical quantization errors by quantizing the UCR time series datasets. Specifically, we linearly offset and rescaled the time series in each dataset such that the minimum and maximum values in any time series correspond to $(0, 255)$ for 8-bit quantization or $(0, 65535)$ for 16-bit quantization. We then obtained the quantized data by applying the floor function to this linearly transformed data.
%  and measuring the noise that this introduces relative to the variance of the data.

To measure the error this introduced, we then inverted the linear transformation and computed the mean squared error between the original and the ``reconstructed'' data. The resulting error values for each dataset, normalized by the dataset's variance, are shown in Figure~\ref{fig:quantize_errs}. These normalized values can be understood as signal-to-noise ratio measurements, where the noise is the quantization error. As the figure illustrates, the quantization error is orders of magnitude smaller than the variance for nearly all datasets, and never worse than $10\times$ smaller, even for 8-bit quantization.

\begin{figure}[h]
\begin{center}
    \includegraphics[width=10cm]{paper/quantize_errs}
    \caption{Quantizing floating-point time series to integers introduces error that is orders of magnitude smaller than the variance of the data. Even with 8 bits, quantization introduces less than 1\% error on 82 of 85 datasets.}
    \label{fig:quantize_errs}
\end{center}
\end{figure}

This of course does not indicate that all time series can be safely quantized. Two counterexamples of which we are aware are 1) timestamps where microsecond or nanosecond resolution matters, and 2) GPS coordinates, where small decimal places may correspond to many meters. However, the above results suggest that quantization is a suitable means of applying \minesp to floating-point data in many applications. This is bolstered by previous work showing that quantization even to a mere six bits \cite{epenthesis} rarely harms classification accuracy, and quantizing to two bits is enough to support many data mining tasks \cite{sax, hotSax, isax, saxvsm}.

% % ------------------------------------------------
% \subsection{Query acceleration}
% % ------------------------------------------------

% \begin{enumerate}
% \item Sliding Mean on 8b data?
% \item Max on 16b data? (Self-driving car acceleration, averaged over two time steps?)
% \item Sliding linear classifier? Examples of action (e.g. "climbing stairs") using MSRC-12 or PAMAP? "Sliding on ice" or something in car data?
% \end{enumerate}

% We can also compare to Sprintz without pushing queries down into the decompress loop (ie, decompress everything first and then query) to show the benefit of our quasi-pushdown; I say ``quasi'' because, except when data gets run-length-encoded, we still do have to decompress it---just not store the decompressed output.

% % ------------------------------------------------
% \subsection{Other Findings}
% % ------------------------------------------------

% We encountered a number of counter-intuitive findings regarding what does and does not improve compression ratio. In the interest of facilitating future research in this area, we briefly describe several of them here.

% 1) Ordering residuals by relative frequency. Absolute value correlates almost perfectly with relative frequency. % Since we'll plot it anyway, point out that residuals usually aren't Laplace distro; more like a power law (in constrast findings in \cite{shorten} for music). Or maybe make that its own bullet.

% 2) Residuals are heavier-tailed than a Laplace distribution, but less heavy-tailed than a power law (in constrast findings in \cite{shorten} for music).

% 3) Nearest-neighbor search. Helped but not worth the bit cost to provide the neighbor index; true for blocks of size 8, 16, 32. (delta then nn) is much bettern than (nn then delta). This is especially interesting given how many motif discovery papers use compression as a heuristic for finding repeating patterns; the disconnect is that they don't include the cost of specifying that a discovered pattern is present.


% % ------------------------------------------------
% \subsection{Discussion}
% % ------------------------------------------------

% Here's a plot of our expected read throughput as a function of number of cores, assuming a 20GB/s memory bandwidth. Is this informative? It's trying to connect it to databases and show that the combination of our high speed and high compression ratio is advantageous. I think it invites questions regarding why we didn't collect real data for this.

% \begin{figure}[h]
% \begin{center}
%     \includegraphics[width=10cm]{paper/expected_thruput}
%     \caption{Expected read throughput on the UCI Gas dataset for various algorithms. Presumably \minesp will do better here since it has the best speed/ratio tradeoff, but who knows since this is currently fake data.}
%     \label{fig:expected_thruput}
% \end{center}
% \end{figure}


% ================================================================
\vspace{3mm}
\section{Related Work} \label{sec:relatedWork}
% ================================================================


% ------------------------------------------------
\subsection{Compression of time series}
% ------------------------------------------------

Most work on compressing time series has focused on lossy techniques. In the data mining literature, Symbolic Aggregate Approximation (SAX) \cite{sax} and its variations \cite{isax, isax2} dominate. These approaches preserve enough information about time series to support specific data mining algorithms (e.g. \cite{fastShapelet, hotSax}), but are extremely lossy; a hundred-sample time series might be compressed into one or two bytes, depending on the exact SAX parameters.

Other lossy approaches include Adaptive Piecewise Constant Approximation \cite{apca}, Piecewise Aggregate Approximation \cite{paa}, and numerous other methods \cite{swab, lemireSegmentation, tsCompressSmartGrid} that approximate time series as sequences of low-order polynomials.

For audio time series specifically, there are a large number of lossy codecs \cite{vorbis, shorten, aac, opus}, as well as a small number of lossless \cite{flac, alac} codecs. In principle, some of these could be applied to non-audio time series. However, modern codecs make such strong assumptions about the possible numbers of channels, sampling rates, bit depths, or other characteristics that it is infeasible to use them on non-audio time series.

Many fewer algorithms exist for lossless time series compression. For floating-point time series, the only algorithm of which we are aware is that of the Gorilla database \cite{gorilla}. This method XORs each value with the previous value to obtain a diff, and then bit packs the diffs. In contrast to our approach, it assumes that time series are univariate and have 64-bit floating-point elements. % The same work also describes a method of compressing integer timestamps. This consists of first delta-delta coding and then applying a similar bit packing compression approach.
% Like other databases \cite{influxDB, berkeleyTreeDB}, they delta-delta code before applying compression %Most time series databases use some form of integer compression (c.f. next section) [], generic predictive coding \cite{akumuli}, or floating-point compression methods.

For lossless compression of integer time series (including timestamps), existing approaches include directly applying general-purpose compressors \cite{respawnDB, openTSDB, chronicleDB, kairosDB, druid}, (double) delta encoding and then applying an integer compressor \cite{influxDB, gorilla}, or predictive coding and byte-packing \cite{akumuli}. These approaches can work well, but tend to offer both less compression and less speed than \mine.

% A final noteworthy method is Blosc \cite{blosc}. While not intended solely for time series (or integers), Blosc's assumption that every $k$ bytes (or bits \cite{bitshuf}) will be correlated for some $k$ is a natural fit for multivariate time series. %grouping of correlated bytes and/or bits makes it well-suited for multivariate time series.


% ------------------------ parquet
%   -currently supports snappy, GZIP, lzo
%   -hdfs also supports these; prolly others also
%   -https://www.cloudera.com/documentation/enterprise/5-6-x/topics/impala_parquet.html#parquet_compression
% ------------------------ Akumuli
%   -DFCM predictor (for both floats and ints?); XOR with prediction, then do something to pack it
%   -"The timestamp can be a simple integer or datetime in ISO 8601 format"
%   -handles int or float values
%   -"This data-structure consists of 4KB blocks"
%   -"They require large amount of memory per data stream to maintain a sliding window of previously seen samples. The larger the context size the better compression ratio gets."
%   -"if we're dealing with 100'000 time-series we'll need about 1GB of memory only for compression contexts."
% ------------------------ RespawnDB
%   -GZIP
%   -"Enabling compression adds a factor of six slowdown in BTDS performance"
% ------------------------ OpenTSDB
%   -LZ0 or snappy for floats, varbyte (1, 2, 4, or 8 bytes) for ints
%   -http://opentsdb.net/faq.html
% ------------------------ Gorilla
%    -custom F64 for values, custom delta-delta for timestamps
% ------------------------ ChronicleDB
%   -lz4
% ------------------------ BerkeleyTreeDB
%   -custom delta-delta
% ------------------------ KairosDB
%     -off-the-shelf compressors: lzo, snappy, probably others
%     -https://github.com/kairosdb/kairosdb/search?utf8=✓&q=compression&type=
% ------------------------ Druid
%   -"The timestamp and metric columns are simple: behind the scenes each of these is an array of integer or floating point values compressed with LZ4."
%       -"metrics" are scalars (the time series)
%   -they also have "dimensions" which are categorical cols; assign each a numeric ID and map IDs to lists of places they occur
%   -http://druid.io/docs/latest/design/segments.html
% ------------------------ InfluxDB
%   -"Timestamp encoding is adaptive and based on the structure of the timestamps that are encoded. It uses a combination of delta encoding, scaling, and compression using simple8b run-length encoding"
%   -"Floats are encoded using an implementation of the Facebook Gorilla paper."
%   -For integers: "If all ZigZag encoded values are less than (1 << 60) - 1, they are compressed using simple8b encoding. If any values are larger than the maximum then all values are stored uncompressed in the block. If all values are identical, run-length encoding is used."
%   -"Strings are encoding using Snappy compression"
%   -and before compaction (where all the above methods get used): "When a write comes in the new points are serialized, compressed using Snappy, and written to a WAL file"
%       -and note: "This means that batching points together is required to achieve high throughput performance. (Optimal batch size seems to be 5,000-10,000 points per batch for many use cases.)"
% ------------------------ LittleTable
%   -LZO1X (LZO) compression http://www.oberhumer.com/opensource/lzo/
% ------------------------ RocksDB (not a ts database, but whatever)
%   -snappy, lz4, zstd


\subsection{Compression of integers}

% Integer compression is not as well-studied as general-purpose compression, but has seen great progress in recent years.

The fastest methods of compressing integers are generally based on bit packing---i.e., using at most $b$ bits to represent values in $\{0, 2^b-1\}$, and storing these bits contiguously \cite{bbp, pfor, fastpfor}. Since $b$ is determined by the largest value that must be encoded, naively applying this method yields limited compression. To improve it, one can encode fixed-size blocks of data at a time, so that $b$ can be set based on the largest values in a block instead of the whole dataset \cite{kGamma, pfor, fastpfor}. A further improvement is to ignore the largest few values when setting $b$ and store their omitted bits separately \cite{pfor, fastpfor}.

\minesp bit packing differs significantly from existing methods in two ways. First, it compresses much smaller blocks of samples. This reduces its throughput as compared to, e.g., \cite{fastpfor}, but significantly improves compression ratios (c.f. Section~\ref{sec:results}). This is because large values only increase $b$ for a few samples instead of for many. Second, \minesp is designed for 8 and 16-bit integers, rather than 32 or 64-bit integers. Existing methods are often inapplicable to lower-bitwidth data (unless converted to higher-bitwidth data) thanks to strong assumptions about bitwidth and data layout.
% $b$ that is even one too large significantly alters the ratio for low-bitwidth data.

A common \cite{flac, shorten} alternative to bit packing is Golomb coding \cite{golomb}, or its special case Rice coding \cite{rice}. The idea is to assume that the values follow a geometric distribution, often with a rate constant fit to the data. %, and therefore make the encoding cost linear in the magnitude of the encoded value.

Both bit packing and Golomb coding are bit-based methods in that they do not guarantee that encoded values will be aligned on byte boundaries. When this is undesirable, one can employ byte-based methods such as 4-Wise Null Suppression \cite{kGamma}, LEB128 \cite{dwarf}, or Varint-G8IU \cite{varintG8IU}. These methods reduce the number of bytes used to store each sample by encoding in a few bits how many bytes are necessary to represent its value, and then encoding only that many bytes. Some, such as Simple8B \cite{simple8b} and SIMD-GroupSimple \cite{groupSimd}, allow fractional bytes to be stored while preserving byte alignment for groups of samples. % These methods allow for efficient universal codes---that is, codes that can represent any possible integer. Universal

% Before applying any of these coding schemes, it is common to apply some transform to the raw data to make the values closer to 0. Such transforms include delta encoding, \cite{fastpfor, bbp}, delta-delta encoding \cite{influxDB}, and linear predictive coding (LPC) \cite{flac}. LPC deterministically generates a prediction for each sample based on the previous samples, and stores the error in the prediction instead of the raw value; when the errors are small, the integers stored are closer to 0. Delta coding and delta-delta coding are special cases wherein each sample is predicted to be the previous sample, or a linear extrapolation from the previous two samples, respectively.

\subsection{General-purpose compression}
While \minesp is not intended to be a general-purpose compression algorithm, a reasonable alternative to using a specialized method would be to apply a general-purpose compression algorithm, possibly after delta coding or other preprocessing. Thanks largely to the development of Asymmetric Numeral Systems (ANS) \cite{ans} for entropy coding, general purpose compressors have advanced greatly in recent years. In particular, Zstd \cite{zstd}, Brotli \cite{brotli}, LZ4 \cite{lz4} and others have attained speed-compression tradeoffs significantly better than traditional methods such as GZIP \cite{gzip}, LZO \cite{lzo}, etc. As described in Section~\ref{sec:intro}, however, these methods have much higher memory requirements that \mine.

% Also of note is Blosc \cite{blosc}, which is especially applicable to multivariate time series as a result of its grouping correlated bits and/or bytes together during preprocessing.

% ------------------------------------------------
\subsection{Predictive Filtering}
% ------------------------------------------------

% TODO move delta, double, LPC descriptions to here. Also talk about \fire.

% Predictive coding in some form is a common element of many compression algorithms. This consists of having some forecaster predict the values of the next byte(s) to be compressed and only storing the prediction error, rather than the original value. When the forecaster is better than chance, the errors will be drawn from a lower-entropy distribution than that of the raw data. In particular, they will often be tightly concentrated around zero \cite{shorten}. The original data can be reconstructed from the errors by having the decoder run the same forecaster and add the encoded errors to its predictions.

For numeric data such as time series, there are four types of predictive coding commonly in use: predictive filtering \cite{png}, delta coding \cite{fastpfor, bbp}, double-delta coding \cite{influxDB, gorilla}, and XOR-based encoding \cite{gorilla}. In predictive filtering, each prediction is a linear combination of a fixed number of recent samples. This can be understood as an autoregressive model or the application of a Finite Impulse Response (FIR) filter. When the filter is learned online, this is termed ``adaptive filtering.''

Delta coding is a special case of predictive filtering where the prediction is always the previous value. Double-delta coding, also called delta-delta coding or delta-of-deltas coding, consists of applying delta coding twice in succession. XOR-based encoding is similar to delta coding, but replaces subtraction of the previous value with the XOR operation. This modification is often desirable for floating-point data \cite{gorilla}.

\fire can be understood as a special case of adaptive filtering. While adaptive filtering is a well-studied mathematical problem in the signal processing literature, we are unaware of a practical algorithm that attains speed within an order of magnitude of that of \justfire. %. or that accounts for the subtleties of low-precision integers.

% Indeed, the only methods

% most similar method is likely that of \cite{neuralBranchPredictor}, in which a

% Moreover, existing databases almost universally use delta, double-delta, or XOR encoding, suggesting that simple  % suggesting that these approaches are the relevant benchmarks.

% Before applying any of these coding schemes, it is common to apply some transform to the raw data to make the values closer to 0. Such transforms include delta encoding, \cite{fastpfor, bbp}, delta-delta encoding \cite{influxDB}, and linear predictive coding (LPC) \cite{flac}. LPC deterministically generates a prediction for each sample based on the previous samples, and stores the error in the prediction instead of the raw value; when the errors are small, the integers stored are closer to 0. Delta coding and delta-delta coding are special cases wherein each sample is predicted to be the previous sample, or a linear extrapolation from the previous two samples, respectively.


% ================================================================
\section{Conclusion} \label{sec:conclusion}
% ================================================================

We introduce \mine, a compression algorithm for multivariate integer time series that achieves state-of-the-art compression ratios across a large number of publicly available datasets. It also attains speeds of up to 3GB/s in a single thread and predictable performance as a function of the number of variables being compressed. Finally, it imposes extremely small memory requirements, making it suitable for compression both at the server and at the network edge. % and server-side compression in the presence of large numbers of streams. % Data stored in \minesp format can be queried at nearly the speed of uncompressed data for even lightweight queries. Moreover,

% As a core component of \minesp, we also introduce the \fire forecaster. \fire achieves throughput of up to 5GB/s while also adapting online to the distribution of the data. \fire can be used as a preprocessor before any compression algorithm and often improves compression ratio dramatically.


% % ================================================================
% \begin{appendix}
% \section{Additional Experiments} \label{sec:moreResults}
% % ================================================================

% \input{appendix.tex}

% \end{appendix}

% ================================================================
% \vspace{-1mm}
% References
% ================================================================

% \nocite{*}

% \IEEEtriggeratref{27}	% trigger column break to make cols even
% \bibliographystyle{ACM-Reference-Format}
\bibliographystyle{abbrv}
\bibliography{doc}

\balance

\end{document}
